% ==========================================================================
% ======================== Dokumentace k PC 1.0 ;-) ========================
% ==========================================================================

% Ověřeno, prošlo bez jediné připomínky - při dodržení následujících pokynů:
% - SPRÁVNĚ vyplnit titulní stranu (!!!)
% - Nadpisy používat v pořadí (zanoření):
% 		- chapter (celá kapitola)
%		- section
% 		- subsection, subsubsection ... 
% - Pro pomlčky V TEXTU používat -- (doslova 2 mínusy). Neplatí pro rozsahy
%	(0-9), spojovníky typu xxx-li apod.
% - Uvozovky - používat české \uv{hezké uvozovky}.
% - Zvýrazňovat \textit{kurzívou}.
% - Názvy modulů, funkcí psát \texttt{neproporcionálním fontem}.
% - URL psát do \url{}
% - In-line matematické výrazy přímo do řádky $uzavřené do dolarů$, složitější
%	na samostatnou řádku $$do dvojitých$$.
% - Pokud např. v zadání zmiňujete LATEX a TEX, použít \latex resp. \tex.

% - Pozor na předložky a spojky - nesmí se vyskytovat na konci řádku => použít
%	nedělitelnou mezeru (~).
% - Nové odstavce dvojitým odřádkováním.

% OBRÁZKY	- Vektorově (!!!) - Illustrator, free Inkscape.
% 			- Rastrově vkládat jen screeny obrazovky.
%			- V textu pak odkazovat pomocí \ref{img1}.
%			- Pro úpravu velikosti změnit scale.
% 	\begin{figure}[h]
%		\centering
%		\includegraphics[scale=1.0]{soubor.pdf/png}
%		\caption{Popisek}
%		\label{img1}
%	\end{figure}

% PŘELOŽENÍ
% - Nainstalovat doporučený MiKTeX.
% - "pdflatex <nazev-souboru>.tex"
% - Pro správné vygenerování odkazů, obsahu nutné vždy spustit 2x (!!!)
% - Pro zjednodušení stačí spouštět přibalený bat, který to udělá sám ;)

% ^^^^^^^^^^^^^^^^^^^^^^^^^^^^^^^^^^^^^^^^^^^^^^^^^^^^^^^^^^^^^^^^^^^^^^^^^^
% ^^^^^^^^^^^^^^^^^ !!!!!!!!!SMAZAT PŘI ODEVZDÁNÍ!!!!!!!!! ^^^^^^^^^^^^^^^^^
% ^^^^^^^^^^^^^^^^^^^^^^^^^^^^^^^^^^^^^^^^^^^^^^^^^^^^^^^^^^^^^^^^^^^^^^^^^^

\documentclass[
	12pt,
	a4paper,
	pdftex,
	czech,
	titlepage
]{report}

% Vkládání formátovaného zdrojového kódu
\usepackage{listings}
\renewcommand{\lstlistingname}{Zdrojový kód}
% Balík listings není standardní a není propojen s babelem.
\renewcommand\lstlistlistingname{Seznam zdrojových kódů}
\lstset{
	backgroundcolor=\color{gray!10},
	basicstyle=\ttfamily,
	columns=fullflexible,
	breakatwhitespace=false,
	breaklines=true,
	captionpos=b,
	commentstyle=\color{green},
	extendedchars=true,
	frame=single,
	keepspaces=true,
	keywordstyle=\color{blue},
	language=c++,
	numbers=left,
	numbersep=5pt,
	numberstyle=\tiny\color{blue},
	rulecolor=\color{gray},
	showspaces=false,
	showtabs=false,
	stringstyle=\color{mymauve},
	tabsize=3,
	title=\lstname{}
}

\usepackage[czech]{babel}
\usepackage[utf8]{inputenc}
\usepackage{lmodern}
\usepackage{textcomp}
\usepackage[T1]{fontenc}
\usepackage{amsfonts}
\usepackage{titlesec}
\usepackage{graphicx}
\usepackage{tikz}
\usetikzlibrary{calc, trees, positioning, arrows, chains, shapes.geometric,
	decorations.pathreplacing, decorations.pathmorphing, shapes, matrix,
	shapes.symbols, circuits, automata, positioning, intersections}

\titleformat{\chapter}
{\normalfont\LARGE\bfseries}{\thechapter}{1em}{}
\titlespacing*{\chapter}{0pt}{0ex plus 1ex minus .2ex}{2.0ex plus .2ex}

\begin{document}

\begin{titlepage}
	\vspace*{-2cm}
	{\centering\includegraphics[scale=1.0]{logo.pdf}\par}
	\centering
	\vspace*{2cm}
	{\Large Semestrální práce z KIV/PC\par}
	\vspace{1.5cm}
	{\Huge\bfseries Přebarvování souvislých oblastí ve snímku\par}
	\vspace{2cm}

	{\Large Jiří Velek\par}
	{\Large A20B0269P\par}
	{\Large jvelek@students.zcu.cz\par}

	\vfill

	{\Large 26.\,10.\,2021}
\end{titlepage}

\tableofcontents
\thispagestyle{empty}
\clearpage

\chapter{Zadání}
\setcounter{page}{1}
Naprogramujte v ANSI C přenositelnou konzolovou aplikaci, která provede v
binárním digitálním obrázku (tj.\ obsahuje jen černé a bílé body) obarvení
souvislých oblastí pomocí níže uvedeného algoritmu Connected-Component Labeling
z oboru počítačového vidění. Vaším úkolem je tedy implementace tohoto algoritmu
a funkcí rozhraní (tj.\ načítání a ukládání obrázku, apod.). Program se bude
spoučtět příkazem

\begin{center}
	\tikzstyle{block} = [rectangle, draw, fill=white!20, rounded corners]
	\tikzstyle{line} = [draw, thick, color=black!50, -latex']
	\begin{tikzpicture}[node distance = 0.5cm, auto]
		\node [block] (1) {
		ccl.exe $\langle$ input-file[.pgm] $\rangle$ $\langle$
		output-file
		$\rangle $
		};
	\end{tikzpicture}
\end{center}

Symbol $\langle$ input-file $\rangle$ zastupuje jméno vstupního souboru s
binárním obrázkem ve formátu Portable Gray Map, přípona souboru nemusí být
uvedena; pokud uvedena není, předpokládejte, že má soubor příponu .pgm. Symbol
$\langle$ output-file $\rangle$ zastupuje jméno výstupního souboru s obarveným
obrázkem, který vytvoří vaše aplikace. Program tedy může být během testování
spuštěn například takto:

\begin{center}
	\tikzstyle{block} = [rectangle, draw, fill=white!20, rounded corners]
	\tikzstyle{line} = [draw, thick, color=black!50, -latex']
	\begin{tikzpicture}[node distance = 0.5cm, auto]
		\node [block] (1) {
			\ldots\textbackslash>ccl.exe
			e:\textbackslash{}images\textbackslash{}img-inp-01.pgm

			e:\textbackslash{}images\textbackslash{}img-res-01.pgm};
	\end{tikzpicture}
\end{center}

Úkolem vašeho programu tedy je vytvořit výsledný soubor s obarveným obrázkem
v~uvedeném umístění a s uvedeným jménem. Vstupní i výstupní obrázek bude uložen
v~souboru ve formátu PGM, který je popsán níže. Obarvení proveďte podle níže
uvedeného algoritmu.
\\

Testujte, zda je vstupní obraz skutečně černobílý. Musí obsahovat pouze pixely
s hodnotou 0x00 a 0xFF\@. Pokud tomu tak není, vypište krátké chybové hlášení
(anglicky) a oznamte chybu operačnímu prostředí pomocí nenulového návratového
kódu.
\\

Hotovou práci odevzdejte v jediném archivu typu ZIP prostřednictvím
automatického odevzdávacího a validačního systému. Postupujte podle instrukcí
uvedených na webu předmětu. Archiv nechť obsahuje všechny zdrojové soubory
potřebné k přeložení programu, makefile pro Windows i Linux(pro překlad v
Linuxu připravte soubor pojmenovaný makefile a pro Windows makefile.win) a
dokumentaci ve formátu PDF vytvořenou v typografickém systému \TeX{}, resp.
\LaTeX{}. Bude-li některá z částí chybět, kontrolní skript Vaši práci odmítne.

\chapter{Analýza úlohy}
Úloha je rozdělena do tří částí: validace a načítání PGM souboru, přebarvení
souvislých komponent, a nakonec zápis do nového PGM souboru.
\section{Načítaní PGM souboru}
PGM soubor je definován následovně:
\begin{enumerate}
	\item řetězec ``P5``
	\item bílý znak
	\item šířka
	\item bílý znak
	\item výška
	\item bílý znak
	\item maximální hodnota šedi, které mohou jednotlivé pixely nabývat
	\item bílý znak
	\item jednotlivé pixely ($šířka * výška$ pixelů)
\end{enumerate}

V našem případě je třeba kontrolovat, zda každý pixel obsahuje hodnotu 0 nebo
255 a pokud ne, program skončí chybovou hláškou. Zároveň je třeba kontrolovat,
zda soubor opravdu obsahuje správný počet pixelů a všechny kontrolní parametry
(šířku, výšku, maximální hodnotu šedi). Po dokončení načítání si program uchová
pouze výšku obrázku, šířku obrázku a sekvenci jednotlivých pixelů.

\section{Přebarvování souvislých komponent}
Problém přebarvování souvislých komponent lze řešit dvěma různými způsoby:
\begin{enumerate}
	\item Pomocí algoritmu DFS
	\item Pomocí algoritmu CCL
\end{enumerate}

\subsection{Řešení pomocí algoritmu Depth-first search}
Algoritmus depth-first search je algoritmus na prohledávání grafu do hloubky.
Samotné řešení našeho problému je rozděleno na dvě funkce: `find\_components' a
`dfs'.

\subsubsection{`find\_components'}
Funkce vytvoří pole `labels' velikosti $šířka * výška$, které značí, zda už se
algoritmus na dané pozici vyskytl, a zároveň si uchová hodnotu oblasti, ve
které se daný pixel nachází.
Potom pro každý pixel, který ještě není v poli, zavolá dfs, a inkrementuje
label, který přiřazuje pixelům.

\subsubsection{`dfs'}
Pokud předaný pixel neobsahuje label, neobsahuje barvu pozadí (černou), a
zároveň jeho souřadnice nejou mimo
hranice velikosti obrázku, pak
funkce nastaví label předanému pixelu a pro všechny jeho sousedy rekurzivně
zavolá sama sebe.

\subsubsection{Komplexita algoritmu}
První cyklus projde přes všechny pixely - $\mathcal{O}(šířka * výška)$,
algoritmus dfs má
komplexitu $\mathcal{O}(V+E)$, kde V je počet pixelů, a E počet hran (každý
pixel má
8 sousedů), tedy $E = 4*šířka*výška-3*šířka-3*výška + 2$, celková komplexita
algoritmu je tedy:
$\mathcal{O}(šířka * výška)$.

\subsection{Řešení pomocí algoritmu Connected component labeling}
Algoritmus Connected component labeling je dvouprůchodový algoritmus. Při
prvním průchodu spojí všechny sousedící vrcholy se stejnou barvou do jedné
množiny a při druhém průchodu tyto množiny očísluje a přidělí barvu jednotlivým
pixelům. Algoritmus má komplexitu
$\mathcal{O}(šířka * výška)$, ale násobící konstanta je nižší, než u
DFS, tedy algoritmus CCL bude o trochu rychlejší.

\chapter{Implementace}
U konkrétní implementace bylo zvoleno řešení algoritmem Connected component
labeling. K realizaci tohoto algoritmu je potřeba vyrobit datovou strukturu
reprezentující množinu. Množina je v programu realizována jako algoritmus
union-find reprezentovaný polem.

\section{Union-find}
Union-find je algoritmus pro reprezentaci množin a k jejich
jednoduchému slučování. Ke sloučení dvou množin $A,B$ stačí pouze najít
``rodičovský vrchol množiny $A$'' a přepsat jeho rodiče na libovolný prvek z
množiny $B$.
\\
\\
Union-find je v programu implementován v souborech \emph{set.h}, \emph{set.c}.

\subsection{Operace find}
Operace find najde ``rodičovský vrchol množiny'', jako parametr přebírá
jakýkoliv prvek množiny. V cyklu prochází pole tak, že: $i = a_i$, dokud platí:
$a_i \neq a_{a_i}$, potom vrátí index $i$.

\subsubsection{Implementace v jazyce C}
\begin{lstlisting}[language=C++, caption={Operace find}, label={source:tbbinit}]
uint8_t find(uint8_t set[], uint8_t element) {
	while(set[element] != element) {
		element = set[element];
	}

	return element;
}
\end{lstlisting}

\subsection{Operace union}
Operace union přebírá dva prvky z různých množin a spojí je do jedné. Nejprve
zavolá find na obou prvcích a potom jednomu z nalezených rodičů nastaví rodiče
na druhého z nalezených.

\subsubsection{Implementace v jazyce C}
\begin{lstlisting}[language=C++, caption={Operace union}, label={source:tbbinit}]
/* funkce je pojmenovana onion, nebot v C je union rezervovane klicove slovo */
void onion(uint8_t set[], uint8_t elementA, uint8_t elementB) {
	uint8_t a = find(set, elementA);
	uint8_t b = find(set, elementB);

	set[a] = b;
}
\end{lstlisting}

\section{CCL}

\subsubsection{Implementace v jazyce C}
\begin{lstlisting}[language=C++, caption={Operace union}, label={source:tbbinit}]
void ccl(uint8_t* data, int width, int height) {
	int y, x;
	uint8_t label = 0;
	for(y = 0; y < height; ++y) {
		for(x = 0; x < width; ++x) {

			if(data[y*width + x] == BACKGROUND) {
				continue;
			}

			

		}
	}
}


\chapter{Uživatelská příručka}

\chapter{Závěr}

\end{document}